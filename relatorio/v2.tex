\chapter{Segunda versão}

A segunda versão do algoritmo é bastante semelhante com a primeira, sendo a
principal diferença o modo de como as arestas são "removidas". Enquanto que a
primeira versão fazia uma cópia do grafo e removia diretamente as arestas, a
segunda versão apenas guarda o \cinline{inDegree} de cada vértice num array, e
depois decrementa o valor presente nesse array. Como tal, as conclusões a tirar
quanto à segunda versão serão também elas bastante semelhantes com as conclusões
da primeira versão.


\section{Análise Informal}
Para a análise informal, realizaremos as experiências
que realizamos na primeira versão, de modo a podermos
facilmente comparar os resultados obtidos.

\begin{table}[H]
	\centering
	\begin{tabular}{| c || c | c | c | c |}
		\hline
		Vértices & Vértices visitados & Arestas visitadas & Comparações & Iterações \\
		\hline\hline
		1024     & 526848             & 1023              & 525824      & 527871    \\
		2048     & 2102272            & 2047              & 2100224     & 2104319   \\
		4096     & 8398848            & 4095              & 8394752     & 8402943   \\
		8192     & 33574912           & 8191              & 33566720    & 33583103  \\
		16384    & 134258688          & 16383             & 134242304   & 134275071 \\
		\hline
	\end{tabular}
	\caption{Resultados da primeira experiência utilizando o segundo algoritmo}
	\label{2-1}
\end{table}

\begin{table}[H]
	\centering
	\begin{tabular}{| c || c | c | c | c |}
		\hline
		Vértices & Vértices visitados & Arestas visitadas & Comparações & Iterações \\
		\hline\hline
		1024     & 526848             & 2044              & 525824      & 528892    \\
		2048     & 2102272            & 4092              & 2100224     & 2106364   \\
		4096     & 8398848            & 8188              & 8394752     & 8407036   \\
		8192     & 33574912           & 16380             & 33566720    & 33591292  \\
		16384    & 134258688          & 32764             & 134242304   & 134291452 \\
		\hline
	\end{tabular}
	\caption{Resultados da segunda experiência utilizando o segundo algoritmo}
	\label{2-2}
\end{table}

Comparando os resultados das listagem \ref{2-1} com \ref{1-2} e da \ref{2-2} com
\ref{1-3}, podemos concluir que de facto o desempenho dos algoritmos nas
métricas selecionadas é bastante semelhante.

Uma vez mais, ao calcular o quociente de $T(2N)$ por $T(N)$, voltamos a obter
aproximadamente 4, resultando numa complexidade quadrática para o algoritmo,
onde o número de arestas presente no grafo tem impacto no número de iterações
realizadas, mas não no número de vértices visitados.

\section{Análise Formal}

O segundo algoritmo tem o mesmo processo da primeira versão, que consiste em
enquanto for possível selecionar um vértice não marcado com \cinline{inDegree}
igual a 0, e iterar as arestas que desse vértice saem. Como tal, o melhor e pior
caso para este segundo algoritmo são iguais aos do primeiro algoritmo com a
exceção de que um termo extra é adicionado para a inicialização do array que
itera pelos vértices todos uma vez.

Como tal, podemos afirmar que o melhor caso é $B(V, E) = 2\norm{V}$, e o pior
caso é $W(V,E) = \frac{\norm{V}^2 + 5\norm{V}}{2} + \norm{E}$, sendo $V$ o
conjunto de vértices do grafo e $E$ o conjunto de arestas. \\
